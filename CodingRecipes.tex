\documentclass[a4paper]{report}
\author{Guillaume Raffy}
\title{Coding Recipes}

\usepackage{bm}
\usepackage{graphicx}
\usepackage{epstopdf}
\usepackage{amsmath,amsfonts,amssymb}
%\usepackage{breqn} % to break equations on multiple lines automatically
\usepackage{amsmath} % provides the ability to break equations on multiple lines through the environment multline
\newcommand{\vect}[1]{\boldsymbol{\mathbf{#1}}}
\newcommand{\mat}[1]{\boldsymbol{\mathbf{#1}}}
\newcommand{\transpose}{^{\top}}
\newcommand{\invtranspose}{^{-\top}}


\begin{document}

\maketitle

	\section{Debugging and profiling techniques}
	
		\subsection{Shared Library interposer (LD_PRELOAD)}
			
			- use examples :
				- intercept malloc calls
				- dirty fixing a bug (transform arguemnt passed to XOpenDisplay as described in http://www.drdobbs.com/building-library-interposers-for-fun-and/184404926)
			

	\section{Understanding -fPIC compiler option}
		
		graffy@physix88:~/bug264/dgemm_bench$ readelf --headers ./dgemm1.exe | grep Type:
		  Type:                              DYN (Shared object file)

		
		
			https://stackoverflow.com/a/30507725/895245
		https://www.collabora.com/about-us/blog/2014/10/01/dynamic-relocs,-runtime-overflows-and-fpic/

		PLT stands for Procedure Linkage Table which is, put simply, used to call external procedures/functions whose address isn't known in the time of linking, and is left to be resolved by the dynamic linker at run time.

		GOT stands for Global Offsets Table and is similarly used to resolve addresses. Both PLT and GOT and other relocation information is explained in greater length in this article.


		excellent article that is short and explains exactly what I want to know : https://www.technovelty.org/linux/plt-and-got-the-key-to-code-sharing-and-dynamic-libraries.html, but the end is a bit short, this helps https://www.slideshare.net/kentarokawamoto/runtime-symbol-resolution.

		got.plt contains pointers to the actual functions but these pointers are initialized in the elf file as stubs (pointers that jump to the execution of the last 2 instructions of each <function-name>@plt)

		Let's consider this call to cblas_dgemm :
			graffy@physix88:~/bug264/dgemm_bench$ readelf --relocs ./dgemm1.exe

			Relocation section '.rela.dyn' at offset 0x638 contains 9 entries:
			  Offset          Info           Type           Sym. Value    Sym. Name + Addend
			000000201d68  000000000008 R_X86_64_RELATIVE                    960
			000000201d70  000000000008 R_X86_64_RELATIVE                    920
			000000202058  000000000008 R_X86_64_RELATIVE                    202058
			000000201fd0  000200000006 R_X86_64_GLOB_DAT 0000000000000000 __cxa_finalize@GLIBC_2.2.5 + 0
			000000201fd8  000500000006 R_X86_64_GLOB_DAT 0000000000000000 _Jv_RegisterClasses + 0
			000000201fe0  000a00000006 R_X86_64_GLOB_DAT 0000000000000000 _ITM_deregisterTMClone + 0
			000000201fe8  000b00000006 R_X86_64_GLOB_DAT 0000000000000000 __libc_start_main@GLIBC_2.2.5 + 0
			000000201ff0  000c00000006 R_X86_64_GLOB_DAT 0000000000000000 __gmon_start__ + 0
			000000201ff8  000d00000006 R_X86_64_GLOB_DAT 0000000000000000 _ITM_registerTMCloneTa + 0

			Relocation section '.rela.plt' at offset 0x710 contains 7 entries:
			  Offset          Info           Type           Sym. Value    Sym. Name + Addend
			000000202018  000100000007 R_X86_64_JUMP_SLO 0000000000000000 printf@GLIBC_2.2.5 + 0
			000000202020  000300000007 R_X86_64_JUMP_SLO 0000000000000000 dsecnd + 0
			000000202028  000400000007 R_X86_64_JUMP_SLO 0000000000000000 MKL_free + 0
			000000202030  000600000007 R_X86_64_JUMP_SLO 0000000000000000 MKL_malloc + 0
			000000202038  000700000007 R_X86_64_JUMP_SLO 0000000000000000 cblas_dgemm + 0
			000000202040  000800000007 R_X86_64_JUMP_SLO 0000000000000000 puts@GLIBC_2.2.5 + 0
			000000202048  000900000007 R_X86_64_JUMP_SLO 0000000000000000 atoi@GLIBC_2.2.5 + 0
		
	
			 bd0:   bf 65 00 00 00          mov    $0x65,%edi
			 bd5:   e8 46 fc ff ff          callq  820 <cblas_dgemm@plt>
			 bda:   48 83 c4 30             add    $0x30,%rsp



			Disassembly of section .plt:

			00000000000007d0 <.plt>:
			 7d0:   ff 35 32 18 20 00       pushq  0x201832(%rip)        # 202008 <_GLOBAL_OFFSET_TABLE_+0x8>
			 7d6:   ff 25 34 18 20 00       jmpq   *0x201834(%rip)        # 202010 <_GLOBAL_OFFSET_TABLE_+0x10>
			 7dc:   0f 1f 40 00             nopl   0x0(%rax)

			00000000000007e0 <printf@plt>:
			 7e0:   ff 25 32 18 20 00       jmpq   *0x201832(%rip)        # 202018 <printf@GLIBC_2.2.5>
			 7e6:   68 00 00 00 00          pushq  $0x0
			 7eb:   e9 e0 ff ff ff          jmpq   7d0 <.plt>

			00000000000007f0 <dsecnd@plt>:
			 7f0:   ff 25 2a 18 20 00       jmpq   *0x20182a(%rip)        # 202020 <dsecnd>
			 7f6:   68 01 00 00 00          pushq  $0x1
			 7fb:   e9 d0 ff ff ff          jmpq   7d0 <.plt>

			0000000000000800 <MKL_free@plt>:
			 800:   ff 25 22 18 20 00       jmpq   *0x201822(%rip)        # 202028 <MKL_free>
			 806:   68 02 00 00 00          pushq  $0x2
			 80b:   e9 c0 ff ff ff          jmpq   7d0 <.plt>

			0000000000000810 <MKL_malloc@plt>:
			 810:   ff 25 1a 18 20 00       jmpq   *0x20181a(%rip)        # 202030 <MKL_malloc>
			 816:   68 03 00 00 00          pushq  $0x3
			 81b:   e9 b0 ff ff ff          jmpq   7d0 <.plt>

			0000000000000820 <cblas_dgemm@plt>:
			 820:   ff 25 12 18 20 00       jmpq   *0x201812(%rip)        # 202038 <cblas_dgemm>
			 826:   68 04 00 00 00          pushq  $0x4
			 82b:   e9 a0 ff ff ff          jmpq   7d0 <.plt>


			Contents of section .got.plt:
			 202000 801d2000 00000000 00000000 00000000  .. .............
			 202010 00000000 00000000 e6070000 00000000  ................
			 202020 f6070000 00000000 06080000 00000000  ................
			 202030 16080000 00000000 26080000 00000000  ........&.......
			 202040 36080000 00000000 46080000 00000000  6.......F.......

		here is the flow of instructions executed :
			1.	bd5:   e8 46 fc ff ff          callq  820 <cblas_dgemm@plt>
			2.  820:   ff 25 12 18 20 00       jmpq   *0x201812(%rip)        # jumps to the value found at 202038 <cblas_dgemm>, which is 0x0000000000000826
			3.  826:   68 04 00 00 00          pushq  $0x4                   # pushes the 2nd argument to _dl_runtime_resolve : index of cblas_dgemm in the GOT
			4.  82b:   e9 a0 ff ff ff          jmpq   7d0 <.plt>
			5.  7d0:   ff 35 32 18 20 00       pushq  0x201832(%rip)         # pushes the 1st argument to _dl_runtime_resolve : A pointer to a structure containing DL info 202008 <_GLOBAL_OFFSET_TABLE_+0x8>
			6.	7d6:   ff 25 34 18 20 00       jmpq   *0x201834(%rip)        # calls _dl_runtime_resolve, which address is located 202010 <_GLOBAL_OFFSET_TABLE_+0x10> (00000000 at init time but filled by linux when loading the elf
			7.  _dl_runtime_resolve resolves the call to cblas_dgemm and puts its actual address <actual_address_of_cblas_dgemm> in the GOT (.got.plt section), at the slot 4 (the index of cblas_dgemm), that is at address 0x202038. Now we consider the address of cblas_dgemm to be cached in 0x202038, so that next call to cblas_dgemm directly finds the address of clas_dgemm instad of using the stub that computes it. 
			8.  _dl_runtime_resolve jumps to <actual_address_of_cblas_dgemm>
	
		Now let's see what's happening on the next call to cblas_dgemm, now that its address is cached :
			9.  bd5:   e8 46 fc ff ff          callq  820 <cblas_dgemm@plt>
			10. 820:   ff 25 12 18 20 00       jmpq   *0x201812(%rip)        # jumps to the value found at 202038 <cblas_dgemm>, which is <actual_address_of_cblas_dgemm>
	

	\section{ImageJ plugins with Eclipse}

	\section{Python}
	
		\subsection{How to deliver python applications ?}



			https://packaging.python.org/wheel_egg/
				Wheel is currently considered the standard for built and binary packaging for Python.

			https://packaging.python.org/distributing/


			example : msspec
			
			[127]graffy@pr079234:~[10:52:34]>sudo port install py-pip
	
			[127]graffy@pr079234:~[10:56:08]>port select --list pip
			Warning: port definitions are more than two weeks old, consider updating them by running 'port selfupdate'.
			Available versions for pip:
				none (active)
				pip27

			[1]graffy@pr079234:~[10:57:30]>sudo port select  --set pip pip27
			Warning: port definitions are more than two weeks old, consider updating them by running 'port selfupdate'.
			Selecting 'pip27' for 'pip' succeeded. 'pip27' is now active.
				
			[OK]graffy@pr079234:~[10:57:41]>which pip
/opt/local/bin/pip


	\section{Eclipse CDT}
	
		debug using gdb

		[127]graffy@pr079234:~/ownCloud/ipr/plasmablob[19:30:43]>sudo port install gdb
		Password:
		--->  Computing dependencies for gdb
		--->  Fetching archive for gdb
		--->  Attempting to fetch gdb-7.12.1_1.darwin_14.x86_64.tbz2 from https://packages.macports.org/gdb
		--->  Attempting to fetch gdb-7.12.1_1.darwin_14.x86_64.tbz2.rmd160 from https://packages.macports.org/gdb
		--->  Installing gdb @7.12.1_1
		--->  Activating gdb @7.12.1_1
		--->  Cleaning gdb
		--->  Updating database of binaries
		--->  Scanning binaries for linking errors
		--->  No broken files found.                             
		--->  Some of the ports you installed have notes:
		  gdb has the following notes:
			You will need to codesign /opt/local/bin/ggdb
	
			See https://sourceware.org/gdb/wiki/BuildingOnDarwin#Giving_gdb_permission_to_control_other_processes
			for more information.
			
		https://sourceware.org/gdb/wiki/BuildingOnDarwin#Giving_gdb_permission_to_control_other_processes
	
		[1]graffy@pr079234:~/build/plasmablob-build/eclipse/plasmablob.project[20:10:57]>sudo launchctl stop com.apple.taskgated

		[OK]graffy@pr079234:~/build/plasmablob-build/eclipse/plasmablob.project[20:11:05]>sudo launchctl start com.apple.taskgated

		after restarting com.apple.taskgated, ggdb managed to run ./plasmablob.exe
		[OK]graffy@pr079234:~/build/plasmablob-build/eclipse/plasmablob.project[20:11:11]>ggdb ./plasmablob.exe 
		GNU gdb (GDB) 7.12.1
		Copyright (C) 2017 Free Software Foundation, Inc.
		License GPLv3+: GNU GPL version 3 or later <http://gnu.org/licenses/gpl.html>
		This is free software: you are free to change and redistribute it.
		There is NO WARRANTY, to the extent permitted by law.  Type "show copying"
		and "show warranty" for details.
		This GDB was configured as "x86_64-apple-darwin14.5.0".
		Type "show configuration" for configuration details.
		For bug reporting instructions, please see:
		<http://www.gnu.org/software/gdb/bugs/>.
		Find the GDB manual and other documentation resources online at:
		<http://www.gnu.org/software/gdb/documentation/>.
		For help, type "help".
		Type "apropos word" to search for commands related to "word"...
		Reading symbols from ./plasmablob.exe...done.
		(gdb) run
		Starting program: /Users/graffy/build/plasmablob-build/eclipse/plasmablob.project/plasmablob.exe 
		
		Version Control Integration in Eclipse (from http://sgpsproject.sourceforge.net/JavierVGomez/index.php/How_to_configure_a_C/C%2B%2B_project_with_Eclipse_and_CMake#Out-Of-Source_Builds)

			Eclipse supports version control systems, e.g. cvs and svn, but for them to work, the project files must be at the root of the source tree. This is not the case with out-of-source builds. The only way to get version control for your project in Eclipse is to have a separate project in the source tree for this purpose. You can either create this project manually or tell CMake to create it for you when creating your project files:

			cmake -G"Eclipse CDT4 - Unix Makefiles" -DMAKE_ECLIPSE_GENERATE_SOURCE_PROJECT=TRUE ../certi_src

			This will create your normal project in the build tree and additionally an extra project in the source tree, we call it the "source-project". In Eclipse you can then import this source-project the same way as you import the normal project. This way you'll have two (or more) projects, one for browsing the sources and doing version control, the other for building your project. 		
		
		
	\subsection{git}
		
		there are tools to configure prompt that help for using git : https://delicious-insights.com/fr/articles/git-submodules/

		\subsubsection{how to handle sub-projects (libraries)}

			apparently, subtrees are better (easier to use) than submodules :
				https://delicious-insights.com/fr/articles/git-submodules/
				https://delicious-insights.com/fr/articles/git-subtrees/
			However, submodules are lighter (subtrees duplicete the module repos in every container).  
			subtrees:
				- duplicate the repos
				+ no specific subtree command to learn, just plain git commands
				+ less error prone (simpler)
			submodules:
				+ sort of symbolic link to mdule repos
				- more error prone (regressions likely)
				- specific submodule commands

			feedback from pnavaro on 26/09/2017 :
				- git submodules doesn't work very well, and he doesn't know about subtrees, but instead he uses cmake's externalproject \ref{sec:cmake-externalproject} feature

			\paragraph{subtree}		
				[OK]graffy@simpa-macbook2:~/ownCloud/ipr/plasmablob[16:44:47]>git fetch ususi
				warning: no common commits
				remote: Counting objects: 15, done.
				remote: Compressing objects: 100% (11/11), done.
				remote: Total 15 (delta 2), reused 0 (delta 0)
				Unpacking objects: 100% (15/15), done.
				From gitlab-ssh.univ-rennes1.fr:graffy/ususi
				 * [new branch]      master     -> ususi/master

				[OK]graffy@simpa-macbook2:~/ownCloud/ipr/plasmablob[16:45:11]>git branch
				* master

				[OK]graffy@simpa-macbook2:~/ownCloud/ipr/plasmablob[17:02:09]>git remote
				origin
				ususi

				[OK]graffy@simpa-macbook2:~/ownCloud/ipr/plasmablob[17:02:27]>git read-tree --prefix=external/ususi -u ususi/master

				[OK]graffy@simpa-macbook2:~/ownCloud/ipr/plasmablob[17:04:41]>ls external/ususi/
				CMakeLists.txt README.rst     include        ususi.cpp

				[OK]graffy@simpa-macbook2:~/ownCloud/ipr/plasmablob[17:04:53]>git status
				On branch master
				Your branch is up-to-date with 'origin/master'.
				Changes to be committed:
				  (use "git reset HEAD <file>..." to unstage)

					new file:   external/ususi/CMakeLists.txt
					new file:   external/ususi/README.rst
					new file:   external/ususi/include/ususi/ususi.hpp
					new file:   external/ususi/ususi.cpp


				[OK]graffy@simpa-macbook2:~/ownCloud/ipr/plasmablob[17:05:03]>git commit
				[master 112aad9] added ususi as a subtree in external/ususi
				 4 files changed, 33 insertions(+)
				 create mode 100644 external/ususi/CMakeLists.txt
				 create mode 100644 external/ususi/README.rst
				 create mode 100644 external/ususi/include/ususi/ususi.hpp
				 create mode 100644 external/ususi/ususi.cpp
					
			\paragraph{git submodules}
		
				\subparagraph{cmake with git submodules or subtrees}
	
					solution 1: use add_external macro
						- http://www.diracprogram.org/doc/release-12/programmers/external_projects.html
						- don't know where to get add_external macro from, so I couldn't test it.

						set(ExternalProjectCMakeArgs
							-DCMAKE_INSTALL_PREFIX=${CMAKE_SOURCE_DIR}/external
							-DCMAKE_Fortran_COMPILER=${CMAKE_Fortran_COMPILER}
							-DCMAKE_C_COMPILER=${CMAKE_C_COMPILER}
							-DCMAKE_CXX_COMPILER=${CMAKE_CXX_COMPILER}
							-DCMAKE_BUILD_TYPE=${CMAKE_BUILD_TYPE}
							-DCBLAS_ROOT=${CBLAS_ROOT}
							-DEIGEN3_ROOT=${EIGEN3_ROOT}
							)

						add_external(pcmsolver)

						Finally we have to make sure that the library is linked.

						If DIRAC f90 modules depend on f90 modules provided by the external library, we have to impose a compilation order:

						add_dependencies(dirac pcmsolver)

					solution 2: manual dependencies
						- simple example in https://github.com/imgix/cmake-submodule-example
						- other example : used in https://github.com/openMVG/openMVG/blob/master/src/CMakeLists.txt with cereal
						add_subdirectory("external/ususi")
						include_directories(external/ususi/include) # not nice as it requires this cmakelists to know where the headers are in the external project...
						add_executable(plasmablob.exe main.cpp radialtransform.cpp util.cpp)
						target_link_libraries(plasmablob.exe ususi ${OpenCV_LIBS})
						add_dependencies(plasmablob.exe ususi)

						- problems :
							- need to call "git submodule init" and "git submodule update"
							- need to know where the location of headers of dependencies inside their project

					solution 3: using cmake's ExternalProject feature
						- using https://cmake.org/cmake/help/v3.0/module/ExternalProject.html
						- example : https://github.com/mfreiholz/cmake-example-external-project
						- problems :
							- need to call "git submodule init" and "git submodule update"

				\subparagraph{gitlab with submodules}
					https://docs.gitlab.com/ce/ci/git_submodules.html#configuring-the-gitmodules-file

			\paragraph{cmake's externalproject feature}
				\label{sec:cmake-externalproject}
				this feature allows cmake to build an external cmake project by pulling it from its git repository. make then pulls the repos before building it. If the user wants to skip the pull stage, he can using an option.
				


		
\end{document}
